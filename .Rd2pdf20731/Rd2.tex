\nonstopmode{}
\documentclass[letterpaper]{book}
\usepackage[times,hyper]{Rd}
\usepackage{makeidx}
\usepackage[utf8]{inputenc} % @SET ENCODING@
% \usepackage{graphicx} % @USE GRAPHICX@
\makeindex{}
\begin{document}
\chapter*{}
\begin{center}
{\textbf{\huge toxvaldbmain}}
\par\bigskip{\large \today}
\end{center}
\inputencoding{utf8}
\ifthenelse{\boolean{Rd@use@hyper}}{\hypersetup{pdftitle = {toxvaldbmain: Builds the ToxValDB V9.x Database}}}{}
\begin{description}
\raggedright{}
\item[Type]\AsIs{Package}
\item[Title]\AsIs{Builds the ToxValDB V9.x Database}
\item[Version]\AsIs{1.0.1}
\item[Author]\AsIs{Taylor Wall}
\item[Maintainer]\AsIs{Taylor Wall }\email{wall.taylor@epa.gov}\AsIs{}
\item[Description]\AsIs{ToxValDB is a database containing quantitative records form in vivo toxicologye studies from 
many sources. The database has 2 main parts - toxval_source containing 
source data in separate tables, and the main toxval schema which combines data from multiple sources 
into a single format. This project moves data from toxval_soruce to toxval.
Data is read from files or other databases into toxval_source and then pulled 
into toxval where terms are converted to standard values. The ToxValDB SOPs describe in more 
detail how to run the code. This version is setup to build ToxValDB v9.5}
\item[Imports]\AsIs{DBI,
RMySQL,
openxlsx,
dplyr,
tidyr,
stringr,
tibble,
janitor,
XML,
logr,
miniUI,
RCurl,
gsubfn,
textclean}
\item[License]\AsIs{MIT + file LICENSE}
\item[Encoding]\AsIs{UTF-8}
\item[LazyData]\AsIs{true}
\item[RoxygenNote]\AsIs{7.2.1}
\item[Suggests]\AsIs{knitr,
rmarkdown}
\item[VignetteBuilder]\AsIs{knitr}
\end{description}
\Rdcontents{\R{} topics documented:}
\inputencoding{utf8}
\HeaderA{cas\_checkSum}{Check CAS RN validity via checksum method}{cas.Rul.checkSum}
\keyword{cas\_functions}{cas\_checkSum}
%
\begin{Description}\relax
For a suspected CAS RN, determine validity by calculating final digit checksum
\end{Description}
%
\begin{Usage}
\begin{verbatim}
cas_checkSum(x, checkLEN = TRUE)
\end{verbatim}
\end{Usage}
%
\begin{Arguments}
\begin{ldescription}
\item[\code{x}] chr. Input vector of values to check. Standard CAS notation using hyphens is fine, as
all non-digit characters are stripped for checksum calculation. Each element of \emph{x} should contain
only one suspected CAS RN to check.

\item[\code{checkLEN}] logi. Should the function check that the non-digit characters of \emph{x} are at least 4, but no
more than 10 digits long? Defaults to TRUE.
\end{ldescription}
\end{Arguments}
%
\begin{Details}\relax
This function performs a very specific type of check for CAS validity, namely whether the final digit checksum follows
the CAS standard. By default, it also ensures that the digit length is compatible with CAS standards. It does nothing
more.

This means that there is no check for valid CAS format. Use the \code{\LinkA{cas\_detect}{cas.Rul.detect}} function to check CAS
format beforehand, or write your own function if necessary.
\end{Details}
%
\begin{Value}
A \code{logical} vector of length \emph{x} denoting whether each \emph{x} is a valid CAS by the checksum method. \code{NA}
input values will remain \code{NA}.
\end{Value}
%
\begin{Note}\relax
This is a vectorized, reasonably high-performance version of the \LinkA{is.cas}{is.cas} function found
in the \LinkA{webchem}{webchem} package. The functionality encompasses only the actual checksum checking of \code{webchem::is.cas};
as mentioned in \code{details}, use \code{\LinkA{cas\_detect}{cas.Rul.detect}} to recreate the CAS format + checksum checking in
\code{webchem::is.cas}. See examples.

Short of looking up against the CAS registry, there is no way to be absolutely sure that even inputs that pass
the checksum test are actually registered CAS RNs. The short digit length of CAS IDs combined with the modulo 10 single-
digit checksum means that even within a set of randomly generated validly-formatted CAS entities, \textasciitilde{}10\% will pass checksum.
\end{Note}
%
\begin{Examples}
\begin{ExampleCode}
cas_good <- c("71-43-2", "18323-44-9", "7732-18-5") #benzene, clindamycin, water
cas_bad  <- c("61-43-2", "18323-40-9", "7732-18-4") #single digit change from good
cas_checkSum(c(cas_good, cas_bad))
\end{ExampleCode}
\end{Examples}
\inputencoding{utf8}
\HeaderA{chem.check}{Check the chemicals from a file Names with special characters are cleaned and trimmed CASRN are fixed (dashes put in, trimmed) and check sums are calculated The output is sent to a file called chemcheck.xlsx in the source data file One option for using this is to edit the source file until no errors are found}{chem.check}
%
\begin{Description}\relax
Check the chemicals from a file
Names with special characters are cleaned and trimmed
CASRN are fixed (dashes put in, trimmed) and check sums are calculated
The output is sent to a file called chemcheck.xlsx in the source data file
One option for using this is to edit the source file until no errors are found
\end{Description}
%
\begin{Usage}
\begin{verbatim}
chem.check(
  res0,
  name.col = "name",
  casrn.col = "casrn",
  source = NULL,
  verbose = F
)
\end{verbatim}
\end{Usage}
%
\begin{Arguments}
\begin{ldescription}
\item[\code{res0}] The data frame in which chemicals names and CASRN will be replaced

\item[\code{name.col}] The column name that contains the chemical names

\item[\code{casrn.col}] The column name that contains the CARN values

\item[\code{source}] The source to be processed. If source=NULL, process all sources

\item[\code{verbose}] If TRUE, print diagnostic messages
\end{ldescription}
\end{Arguments}
%
\begin{Value}
Return a list with fixed CASRN and name and flags indicating if fixes were made:
res0=res0,name.OK=name.OK,casrn.OK=casrn.OK,checksum.OK=checksum.OK
\end{Value}
\inputencoding{utf8}
\HeaderA{chem.check.v2}{Check the chemicals from a file Names with special characters are cleaned and trimmed CASRN are fixed (dashes put in, trimmed) and check sums are calculated The output is sent to a file called chemcheck.xlsx in the source data file One option for using this is to edit the source file until no errors are found}{chem.check.v2}
%
\begin{Description}\relax
Check the chemicals from a file
Names with special characters are cleaned and trimmed
CASRN are fixed (dashes put in, trimmed) and check sums are calculated
The output is sent to a file called chemcheck.xlsx in the source data file
One option for using this is to edit the source file until no errors are found
\end{Description}
%
\begin{Usage}
\begin{verbatim}
chem.check.v2(res0, source = NULL, verbose = F)
\end{verbatim}
\end{Usage}
%
\begin{Arguments}
\begin{ldescription}
\item[\code{res0}] The data frame in which chemicals names and CASRN will be replaced

\item[\code{source}] The source to be processed. If source=NULL, process all sources

\item[\code{verbose}] If TRUE, print diagnostic messages
\end{ldescription}
\end{Arguments}
%
\begin{Value}
Return a list with fixed CASRN and name and flags indicating if fixes were made:
res0=res0,name.OK=name.OK,casrn.OK=casrn.OK,checksum.OK=checksum.OK
\end{Value}
\inputencoding{utf8}
\HeaderA{clean.last.character}{Clean unneeded characters from the end of a string}{clean.last.character}
%
\begin{Description}\relax
Clean unneeded characters from the end of a string
\end{Description}
%
\begin{Usage}
\begin{verbatim}
clean.last.character(x)
\end{verbatim}
\end{Usage}
%
\begin{Arguments}
\begin{ldescription}
\item[\code{x}] String to be cleaned
\end{ldescription}
\end{Arguments}
%
\begin{Value}
The cleaned string
\end{Value}
\inputencoding{utf8}
\HeaderA{clean.toxval.by.source}{Delete a portion of the contents of the toxval database}{clean.toxval.by.source}
%
\begin{Description}\relax
Delete a portion of the contents of the toxval database
\end{Description}
%
\begin{Usage}
\begin{verbatim}
clean.toxval.by.source(toxval.db, source)
\end{verbatim}
\end{Usage}
%
\begin{Arguments}
\begin{ldescription}
\item[\code{toxval.db}] The version of toxval from which the data is deleted.

\item[\code{source}] The data source name
\end{ldescription}
\end{Arguments}
%
\begin{Value}
The database will be altered
\end{Value}
\inputencoding{utf8}
\HeaderA{contains}{Find out if one string contains another}{contains}
%
\begin{Description}\relax
Find out if one string contains another
\end{Description}
%
\begin{Usage}
\begin{verbatim}
contains(x, query, verbose = F)
\end{verbatim}
\end{Usage}
%
\begin{Arguments}
\begin{ldescription}
\item[\code{x}] The string to be searched in

\item[\code{query}] the second string

\item[\code{verbose}] if TRUE, the two strings are printed
\end{ldescription}
\end{Arguments}
%
\begin{Value}
if x contains query, return TRUE, FALSE otherwise
\end{Value}
\inputencoding{utf8}
\HeaderA{count.source\_hash}{Look for duplicated source\_hash}{count.source.Rul.hash}
%
\begin{Description}\relax
Look for duplicated source\_hash
\end{Description}
%
\begin{Usage}
\begin{verbatim}
## S3 method for class 'source_hash'
count(toxval.db)
\end{verbatim}
\end{Usage}
%
\begin{Arguments}
\begin{ldescription}
\item[\code{toxval.db}] The version of toxval in which the data is altered.
\end{ldescription}
\end{Arguments}
\inputencoding{utf8}
\HeaderA{export.all.by.source}{Build a data frame of the data from toxval and export by source as a series of xlsx files}{export.all.by.source}
%
\begin{Description}\relax
Build a data frame of the data from toxval and export by source as a
series of xlsx files
\end{Description}
%
\begin{Usage}
\begin{verbatim}
export.all.by.source(toxval.db, source = NULL)
\end{verbatim}
\end{Usage}
%
\begin{Arguments}
\begin{ldescription}
\item[\code{toxval.db}] Database version

\item[\code{source}] The source to be updated
\#' @return for each source writes an Excel file with the name
../export/export\_by\_source\_data/toxval\_all\_toxval.db\_source.xlsx
\end{ldescription}
\end{Arguments}
\inputencoding{utf8}
\HeaderA{export.all.with.references.v93}{Build a data frame of the PODs and exports as xlsx}{export.all.with.references.v93}
%
\begin{Description}\relax
Build a data frame of the PODs and exports as xlsx
\end{Description}
%
\begin{Usage}
\begin{verbatim}
export.all.with.references.v93(
  toxval.db = "res_toxval_v94",
  file.name = NA,
  pfas = T
)
\end{verbatim}
\end{Usage}
%
\begin{Arguments}
\begin{ldescription}
\item[\code{toxval.db}] Database version

\item[\code{file.name}] If not NA, this is a file containing chemicals, and only those chemicals will be exported

\item[\code{human\_eco}] Either 'human health' or 'eco'
\end{ldescription}
\end{Arguments}
%
\begin{Value}
writes an Excel file with the name
../export/toxval\_pod\_summary\_[human\_eco]\_Sys.Date().xlsx
\end{Value}
\inputencoding{utf8}
\HeaderA{export.for.oppt}{Build a data frame of the PODs and exports as xlsx}{export.for.oppt}
%
\begin{Description}\relax
Build a data frame of the PODs and exports as xlsx
\end{Description}
%
\begin{Usage}
\begin{verbatim}
export.for.oppt(toxval.db = "res_toxval_v94", file.name = "TSCA PICS")
\end{verbatim}
\end{Usage}
%
\begin{Arguments}
\begin{ldescription}
\item[\code{toxval.db}] Database version

\item[\code{file.name}] If not NA, this is a file containing chemicals, and only those chemicals will be exported

\item[\code{human\_eco}] Either 'human health' or 'eco'
\end{ldescription}
\end{Arguments}
%
\begin{Value}
writes an Excel file with the name
../export/toxval\_pod\_summary\_[human\_eco]\_Sys.Date().xlsx
\end{Value}
\inputencoding{utf8}
\HeaderA{export.for.study\_type}{Export data required for setting the study type}{export.for.study.Rul.type}
%
\begin{Description}\relax
Export data required for setting the study type
\end{Description}
%
\begin{Usage}
\begin{verbatim}
export.for.study_type(toxval.db, source = NULL)
\end{verbatim}
\end{Usage}
%
\begin{Arguments}
\begin{ldescription}
\item[\code{toxval.db}] Database version
\end{ldescription}
\end{Arguments}
%
\begin{Value}
writes an Excel file with the name
../export/toxval\_pod\_summary\_[human\_eco]\_Sys.Date().xlsx
\end{Value}
\inputencoding{utf8}
\HeaderA{export.for.toxvaldb.manuscript}{Build a data frame of the data for the toxval manuscript}{export.for.toxvaldb.manuscript}
%
\begin{Description}\relax
Build a data frame of the data for the toxval manuscript
\end{Description}
%
\begin{Usage}
\begin{verbatim}
export.for.toxvaldb.manuscript(toxval.db)
\end{verbatim}
\end{Usage}
%
\begin{Arguments}
\begin{ldescription}
\item[\code{toxval.db}] Database version

\item[\code{source}] The source to be updated
\end{ldescription}
\end{Arguments}
%
\begin{Value}
Write a file with the results
\end{Value}
\inputencoding{utf8}
\HeaderA{export.missing.dictionary.entries}{Find "original" values that have not been included in the dictionaries and export them}{export.missing.dictionary.entries}
%
\begin{Description}\relax
Find "original" values that have not been included in the dictionaries
and export them
\end{Description}
%
\begin{Usage}
\begin{verbatim}
export.missing.dictionary.entries(toxval.db, source = NULL, subsource = NULL)
\end{verbatim}
\end{Usage}
%
\begin{Arguments}
\begin{ldescription}
\item[\code{toxval.db}] The version of toxval in which the data is altered.

\item[\code{source}] The source to be fixed. If source=NULL, fix all sources
\end{ldescription}
\end{Arguments}
%
\begin{Value}
An excel file in dictionaries with the missing entries
"missing dictionary entries Sys.Date.xlsx"
\end{Value}
\inputencoding{utf8}
\HeaderA{export.missing.strains}{Set the strain information in toxval}{export.missing.strains}
%
\begin{Description}\relax
Set the strain information in toxval
\end{Description}
%
\begin{Usage}
\begin{verbatim}
export.missing.strains(toxval.db, date_string = "2022-08-15")
\end{verbatim}
\end{Usage}
%
\begin{Arguments}
\begin{ldescription}
\item[\code{toxval.db}] The version of the database to use

\item[\code{date\_string}] The date of the latest dictinary version
\end{ldescription}
\end{Arguments}
\inputencoding{utf8}
\HeaderA{export.missing.toxval\_type}{Export any toxval\_types that are not in the toxval\_type dictionary}{export.missing.toxval.Rul.type}
%
\begin{Description}\relax
Export any toxval\_types that are not in the toxval\_type dictionary
\end{Description}
%
\begin{Usage}
\begin{verbatim}
export.missing.toxval_type(toxval.db)
\end{verbatim}
\end{Usage}
%
\begin{Arguments}
\begin{ldescription}
\item[\code{toxval.db}] The version of toxval in which the data is altered.
\end{ldescription}
\end{Arguments}
%
\begin{Value}
An excel file in dictionaries with the missing entries
"dictionary/missing/missing\_toxval\_type Sys.Date.xlsx"
\end{Value}
\inputencoding{utf8}
\HeaderA{export.toxvaldb.subset}{Build a data frame of the data from toxval for a subset of chemicals}{export.toxvaldb.subset}
%
\begin{Description}\relax
Build a data frame of the data from toxval for a subset of chemicals
\end{Description}
%
\begin{Usage}
\begin{verbatim}
export.toxvaldb.subset(toxval.db, filename)
\end{verbatim}
\end{Usage}
%
\begin{Arguments}
\begin{ldescription}
\item[\code{toxval.db}] Database version

\item[\code{filename}] The name of the file the be imported - should be a short name that
will be used in the output filename. This is an xlsx file and needs a column labeled dtxsid
\end{ldescription}
\end{Arguments}
%
\begin{Value}
Write a file with the results
\end{Value}
\inputencoding{utf8}
\HeaderA{fill.chemical.by.source}{Fill the chemical table}{fill.chemical.by.source}
%
\begin{Description}\relax
Fill the chemical table
\end{Description}
%
\begin{Usage}
\begin{verbatim}
## S3 method for class 'chemical.by.source'
fill(toxval.db, source, verbose = T)
\end{verbatim}
\end{Usage}
%
\begin{Arguments}
\begin{ldescription}
\item[\code{toxval.db}] The version of toxvaldb to use.

\item[\code{source}] The source to be used

\item[\code{verbose}] If TRUE, print out extra diagnostic messages
\end{ldescription}
\end{Arguments}
\inputencoding{utf8}
\HeaderA{fill.toxval.defaults}{Set Toxval Defaults}{fill.toxval.defaults}
%
\begin{Description}\relax
Set Toxval Defaults
\end{Description}
%
\begin{Usage}
\begin{verbatim}
## S3 method for class 'toxval.defaults'
fill(toxval.db, mat)
\end{verbatim}
\end{Usage}
%
\begin{Arguments}
\begin{ldescription}
\item[\code{toxval.db}] The version of toxval from which to set defaults.

\item[\code{mat}] An input matrix of data
\end{ldescription}
\end{Arguments}
%
\begin{Value}
The data matrix afer fixing
\end{Value}
\inputencoding{utf8}
\HeaderA{fill.toxval.defaults.global.by.source}{Set Toxval Defaults globally,  replacing blanks with -}{fill.toxval.defaults.global.by.source}
%
\begin{Description}\relax
Set Toxval Defaults globally,  replacing blanks with -
\end{Description}
%
\begin{Usage}
\begin{verbatim}
## S3 method for class 'toxval.defaults.global.by.source'
fill(toxval.db, source = NULL)
\end{verbatim}
\end{Usage}
%
\begin{Arguments}
\begin{ldescription}
\item[\code{toxval.db}] The version of toxval from which to set defaults.

\item[\code{source}] The source to be fixed
\end{ldescription}
\end{Arguments}
\inputencoding{utf8}
\HeaderA{fix.all.param.by.source}{Alter the contents of toxval according to an excel dictionary file with fields - exposure\_method, exposure\_route, sex,strain, study\_duration\_class, study\_duration\_units, study\_type, toxval\_type, exposure\_form, media, toxval\_subtype}{fix.all.param.by.source}
%
\begin{Description}\relax
Alter the contents of toxval according to an excel dictionary file with fields -
exposure\_method, exposure\_route, sex,strain, study\_duration\_class, study\_duration\_units, study\_type,
toxval\_type, exposure\_form, media, toxval\_subtype
\end{Description}
%
\begin{Usage}
\begin{verbatim}
fix.all.param.by.source(
  toxval.db,
  source = NULL,
  subsource = NULL,
  fill.toxval_fix = T
)
\end{verbatim}
\end{Usage}
%
\begin{Arguments}
\begin{ldescription}
\item[\code{toxval.db}] The version of toxval in which the data is altered.

\item[\code{source}] The source to be fixed. If source=NULL, fix all sources

\item[\code{fill.toxval\_fix}] If TRUE (default) read the dictionaries into the toxval\_fix table
\end{ldescription}
\end{Arguments}
%
\begin{Value}
The database will be altered
\end{Value}
\inputencoding{utf8}
\HeaderA{fix.casrn}{Fix a CASRN that has one of several problems}{fix.casrn}
%
\begin{Description}\relax
Fix a CASRN that has one of several problems
\end{Description}
%
\begin{Usage}
\begin{verbatim}
fix.casrn(casrn, cname = "", verbose = F)
\end{verbatim}
\end{Usage}
%
\begin{Arguments}
\begin{ldescription}
\item[\code{casrn}] Input CASRN to be fixed

\item[\code{cname}] An optional chemical name

\item[\code{verbose}] if TRUE, print hte input values
\end{ldescription}
\end{Arguments}
%
\begin{Value}
the fixed CASRN
\end{Value}
\inputencoding{utf8}
\HeaderA{fix.critical\_effect.icf.by.source}{standardize critical\_effect in toxval table based on icf dictionary and toxval critical effects dictionary}{fix.critical.Rul.effect.icf.by.source}
%
\begin{Description}\relax
standardize critical\_effect in toxval table based on icf dictionary and toxval critical effects dictionary
\end{Description}
%
\begin{Usage}
\begin{verbatim}
fix.critical_effect.icf.by.source(toxval.db, source)
\end{verbatim}
\end{Usage}
%
\begin{Arguments}
\begin{ldescription}
\item[\code{toxval.db}] The version of toxvaldb to use.

\item[\code{source}] THe source to be fixed
\end{ldescription}
\end{Arguments}
\inputencoding{utf8}
\HeaderA{fix.empty.by.source}{Set all empty cells in toxval to '-'}{fix.empty.by.source}
%
\begin{Description}\relax
Set all empty cells in toxval to '-'
\end{Description}
%
\begin{Usage}
\begin{verbatim}
fix.empty.by.source(toxval.db, source = NULL)
\end{verbatim}
\end{Usage}
%
\begin{Arguments}
\begin{ldescription}
\item[\code{toxval.db}] The version of toxval in which the data is altered.

\item[\code{source}] The source to be fixed
\end{ldescription}
\end{Arguments}
%
\begin{Value}
The database will be altered
\end{Value}
\inputencoding{utf8}
\HeaderA{fix.empty.record\_source.by.source}{Set all empty cells in record\_source to '-'}{fix.empty.record.Rul.source.by.source}
%
\begin{Description}\relax
Set all empty cells in record\_source to '-'
\end{Description}
%
\begin{Usage}
\begin{verbatim}
fix.empty.record_source.by.source(toxval.db, source = NULL)
\end{verbatim}
\end{Usage}
%
\begin{Arguments}
\begin{ldescription}
\item[\code{toxval.db}] The version of toxval in which the data is altered.

\item[\code{source}] The source to be fixed
\end{ldescription}
\end{Arguments}
%
\begin{Value}
The database will be altered
\end{Value}
\inputencoding{utf8}
\HeaderA{fix.exposure.params}{Fix the exposure fields: exposure\_method, exposure\_route, exposure\_form based on a 3 column dictionary \textasciitilde{}/dictionary/exposure\_route\_method\_form.xlsx}{fix.exposure.params}
%
\begin{Description}\relax
Fix the exposure fields: exposure\_method, exposure\_route, exposure\_form
based on a 3 column dictionary \textasciitilde{}/dictionary/exposure\_route\_method\_form.xlsx
\end{Description}
%
\begin{Usage}
\begin{verbatim}
fix.exposure.params(toxval.db, source = NULL, subsource = NULL)
\end{verbatim}
\end{Usage}
%
\begin{Arguments}
\begin{ldescription}
\item[\code{toxval.db}] The version of toxval in which the data is altered.

\item[\code{source}] The source to be fixed. If source=NULL, fix all sources

\item[\code{fill.toxval\_fix}] If TRUE (default) read the dictionaries into the toxval\_fix table
\end{ldescription}
\end{Arguments}
%
\begin{Value}
The database will be altered
\end{Value}
\inputencoding{utf8}
\HeaderA{fix.exposure\_method.and.form.by.source}{Update the exposure route, method and form from a dictionary}{fix.exposure.Rul.method.and.form.by.source}
%
\begin{Description}\relax
Update the exposure route, method and form from a dictionary
\end{Description}
%
\begin{Usage}
\begin{verbatim}
fix.exposure_method.and.form.by.source(toxval.db, source)
\end{verbatim}
\end{Usage}
%
\begin{Arguments}
\begin{ldescription}
\item[\code{toxval.db}] The database version to use

\item[\code{source}] The source to process
\end{ldescription}
\end{Arguments}
\inputencoding{utf8}
\HeaderA{fix.generation.by.source}{Alter the contents of toxval according to an excel dictionary file with field generation}{fix.generation.by.source}
%
\begin{Description}\relax
Alter the contents of toxval according to an excel dictionary file with field generation
\end{Description}
%
\begin{Usage}
\begin{verbatim}
fix.generation.by.source(toxval.db, source)
\end{verbatim}
\end{Usage}
%
\begin{Arguments}
\begin{ldescription}
\item[\code{toxval.db}] The version of toxval in which the data is altered.

\item[\code{source}] The source to be processes
\end{ldescription}
\end{Arguments}
%
\begin{Value}
The database will be altered
\end{Value}
\inputencoding{utf8}
\HeaderA{fix.human\_eco.by.source}{Fix the human\_eco flag}{fix.human.Rul.eco.by.source}
%
\begin{Description}\relax
Fix the human\_eco flag
\end{Description}
%
\begin{Usage}
\begin{verbatim}
fix.human_eco.by.source(toxval.db, source = NULL, reset = T)
\end{verbatim}
\end{Usage}
%
\begin{Arguments}
\begin{ldescription}
\item[\code{toxval.db}] The version of toxval in which the data is altered.

\item[\code{source}] The source to be fixed. If NULL, fix all sources

\item[\code{reset}] If TRUE, reset all values to 'not specified' before processing all records in the source
\end{ldescription}
\end{Arguments}
%
\begin{Value}
The database will be altered
\end{Value}
\inputencoding{utf8}
\HeaderA{fix.non\_ascii.v2}{Flag and fix non-ascii characters in the database}{fix.non.Rul.ascii.v2}
%
\begin{Description}\relax
Flag and fix non-ascii characters in the database
\end{Description}
%
\begin{Usage}
\begin{verbatim}
fix.non_ascii.v2(df, source)
\end{verbatim}
\end{Usage}
%
\begin{Arguments}
\begin{ldescription}
\item[\code{df}] The dataframe to be processed

\item[\code{The}] source to be fixed
\end{ldescription}
\end{Arguments}
%
\begin{Value}
The dataframe with non ascii characters replaced with cleaned versions
\end{Value}
\inputencoding{utf8}
\HeaderA{fix.priority\_id.by.source}{Fix the priority\_id in the toxval table based on source}{fix.priority.Rul.id.by.source}
%
\begin{Description}\relax
Fix the priority\_id in the toxval table based on source
\end{Description}
%
\begin{Usage}
\begin{verbatim}
fix.priority_id.by.source(toxval.db, source = NULL)
\end{verbatim}
\end{Usage}
%
\begin{Arguments}
\begin{ldescription}
\item[\code{toxval.db}] The version of toxvaldb to use.

\item[\code{source}] The source to be fixed, If NULL, set for all sources
\end{ldescription}
\end{Arguments}
\inputencoding{utf8}
\HeaderA{fix.qc\_status.by.source}{Fix the qa\_status flag}{fix.qc.Rul.status.by.source}
%
\begin{Description}\relax
Fix the qa\_status flag
\end{Description}
%
\begin{Usage}
\begin{verbatim}
fix.qc_status.by.source(toxval.db, source = NULL, reset = T)
\end{verbatim}
\end{Usage}
%
\begin{Arguments}
\begin{ldescription}
\item[\code{toxval.db}] The version of toxval in which the data is altered.

\item[\code{source}] The source to be fixed

\item[\code{reset}] If TRUE, reset all values to 'pass' before setting
\end{ldescription}
\end{Arguments}
%
\begin{Value}
The database will be altered
\end{Value}
\inputencoding{utf8}
\HeaderA{fix.risk\_assessment\_class.by.source}{Set the risk assessment class of toxval according to an excel dictionary. Values may beset multiple times, so the excel sheet should be ordered so that the last ones to be set are last}{fix.risk.Rul.assessment.Rul.class.by.source}
%
\begin{Description}\relax
Set the risk assessment class of toxval according to an excel dictionary.
Values may beset multiple times, so the excel sheet should be ordered so that
the last ones to be set are last
\end{Description}
%
\begin{Usage}
\begin{verbatim}
fix.risk_assessment_class.by.source(toxval.db, source = NULL, restart = T)
\end{verbatim}
\end{Usage}
%
\begin{Arguments}
\begin{ldescription}
\item[\code{toxval.db}] The version of toxval in which the data is altered.

\item[\code{source}] The source to be updated

\item[\code{restart}] If TRUE, delete all values and start from scratch
\end{ldescription}
\end{Arguments}
\inputencoding{utf8}
\HeaderA{fix.single.param.by.source}{Alter the contents of toxval according to an excel dictionary}{fix.single.param.by.source}
%
\begin{Description}\relax
Alter the contents of toxval according to an excel dictionary
\end{Description}
%
\begin{Usage}
\begin{verbatim}
fix.single.param.by.source(toxval.db, param, source, ignore = FALSE)
\end{verbatim}
\end{Usage}
%
\begin{Arguments}
\begin{ldescription}
\item[\code{toxval.db}] The version of toxval in which the data is altered.

\item[\code{param}] The parameter value to be fixed

\item[\code{source}] The source to be fixed

\item[\code{ignore}] If TRUE allow missing values to be ignored
\end{ldescription}
\end{Arguments}
%
\begin{Value}
The database will be altered
\end{Value}
\inputencoding{utf8}
\HeaderA{fix.species.v2}{Set the species\_id column in toxval}{fix.species.v2}
%
\begin{Description}\relax
This function replaces fix.species
This function precedes toxvaldb.load.species
\end{Description}
%
\begin{Usage}
\begin{verbatim}
fix.species.v2(toxval.db, source, date_string = "2023-02-14")
\end{verbatim}
\end{Usage}
%
\begin{Arguments}
\begin{ldescription}
\item[\code{toxval.db}] The version of the database to use

\item[\code{source}] The source to be fixed

\item[\code{date\_string}] The date version of the dictionary
\end{ldescription}
\end{Arguments}
\inputencoding{utf8}
\HeaderA{fix.strain.v2}{Set the strain information in toxval}{fix.strain.v2}
%
\begin{Description}\relax
Set the strain information in toxval
\end{Description}
%
\begin{Usage}
\begin{verbatim}
fix.strain.v2(toxval.db, source = NULL, date_string = "2023-04-03")
\end{verbatim}
\end{Usage}
%
\begin{Arguments}
\begin{ldescription}
\item[\code{toxval.db}] The version of the database to use

\item[\code{source}] The source to be fixed. If NULL, fix for all sources

\item[\code{date\_string}] The date of the latest dictionary version
\end{ldescription}
\end{Arguments}
\inputencoding{utf8}
\HeaderA{fix.study\_duration.params}{Fix the study duration fields: study\_duration\_value, study\_duration\_units, study\_duration\_class based on a 3 column dictionary \textasciitilde{}/dictionary/exposure\_route\_method\_form.xlsx}{fix.study.Rul.duration.params}
%
\begin{Description}\relax
Fix the study duration fields: study\_duration\_value, study\_duration\_units, study\_duration\_class
based on a 3 column dictionary \textasciitilde{}/dictionary/exposure\_route\_method\_form.xlsx
\end{Description}
%
\begin{Usage}
\begin{verbatim}
fix.study_duration.params(toxval.db, source = NULL, subsource = NULL)
\end{verbatim}
\end{Usage}
%
\begin{Arguments}
\begin{ldescription}
\item[\code{toxval.db}] The version of toxval in which the data is altered.

\item[\code{source}] The source to be fixed. If source=NULL, fix all sources

\item[\code{fill.toxval\_fix}] If TRUE (default) read the dictionaries into the toxval\_fix table
\end{ldescription}
\end{Arguments}
%
\begin{Value}
The database will be altered
\end{Value}
\inputencoding{utf8}
\HeaderA{fix.study\_type.manual}{Fix the study\_type using manual curation}{fix.study.Rul.type.manual}
%
\begin{Description}\relax
Fix the study\_type using manual curation
\end{Description}
%
\begin{Usage}
\begin{verbatim}
fix.study_type.manual(toxval.db, source = NULL, sys.date = "2023-04-10")
\end{verbatim}
\end{Usage}
%
\begin{Arguments}
\begin{ldescription}
\item[\code{toxval.db}] The version of toxval in which the data is altered.
\end{ldescription}
\end{Arguments}
%
\begin{Value}
The database will be altered
\end{Value}
\inputencoding{utf8}
\HeaderA{fix.trim\_spaces}{Trim leading and trailing blanks from all character columns}{fix.trim.Rul.spaces}
%
\begin{Description}\relax
Trim leading and trailing blanks from all character columns
\end{Description}
%
\begin{Usage}
\begin{verbatim}
fix.trim_spaces(res)
\end{verbatim}
\end{Usage}
%
\begin{Arguments}
\begin{ldescription}
\item[\code{toxval.db}] The version of toxval in which the data is altered.

\item[\code{source}] The source to be fixed. If source=NULL, fix all sources

\item[\code{fill.toxval\_fix}] If TRUE (default) read the dictionaries into the toxval\_fix table
\end{ldescription}
\end{Arguments}
%
\begin{Value}
The database will be altered
\end{Value}
\inputencoding{utf8}
\HeaderA{fix.units.by.source}{Do all of the fixes to units}{fix.units.by.source}
%
\begin{Description}\relax
\begin{enumerate}

\item{} All of these steps operate on the toxval\_units column.
\item{} Replace variant unit names with standard ones, running fix.single.param.new.by.source.R
This fixes issues like variant names for mg/kg-day and uses the dictionary
file dictionary/toxval\_units\_5.xlsx
\item{} Fix special characters in toxval\_units
\item{} Fix issues with units containing extra characters for some ECOTOX records
\item{} Convert units that are multiples of standard ones (e.g. ppb to ppm). This
uses the dictionary file dictionary/toxval\_units conversions 2018-09-12.xlsx
\item{} Run conversions from molar to mg units, using MW. This uses the dictionary file
dictionary/MW conversions.xlsx
\item{} Convert ppm to mg/m3 for inhalation studies. This uses the conversion Concentration
(mg/m3) = 0.0409 x concentration (ppm) x molecular weight. See
https://cfpub.epa.gov/ncer\_abstracts/index.cfm/fuseaction/display.files/fileID/14285.
This function requires htat the DSSTox external chemical\_id be set
\item{} Convert ppm to mg/kg-day in toxval according to a species-specific
conversion factor for oral exposures. This uses the dictionary file
dictionary/ppm to mgkgday by animal.xlsx
See: www10.plala.or.jp/biostatistics/1-3.doc
This probbaly assumes feed rather than water
\item{} Make sure that eco studies are in mg/L and human health in mg/m3

\end{enumerate}

\end{Description}
%
\begin{Usage}
\begin{verbatim}
fix.units.by.source(
  toxval.db,
  source = NULL,
  subsource = NULL,
  do.convert.units = F
)
\end{verbatim}
\end{Usage}
%
\begin{Arguments}
\begin{ldescription}
\item[\code{toxval.db}] The version of toxvaldb to use.

\item[\code{source}] Source to be fixed

\item[\code{do.convert.units}] If TRUE, so unit conversions, as opposed to just cleaning
\end{ldescription}
\end{Arguments}
\inputencoding{utf8}
\HeaderA{generate.originals}{Duplicate any columns with '\_original' Set Toxval Defaults}{generate.originals}
%
\begin{Description}\relax
Duplicate any columns with '\_original'
Set Toxval Defaults
\end{Description}
%
\begin{Usage}
\begin{verbatim}
generate.originals(toxval.db, mat)
\end{verbatim}
\end{Usage}
%
\begin{Arguments}
\begin{ldescription}
\item[\code{toxval.db}] The version of toxval from which to set defaults.

\item[\code{mat}] THe matrix of data to be altered
\end{ldescription}
\end{Arguments}
%
\begin{Value}
The altered input matrix
\end{Value}
\inputencoding{utf8}
\HeaderA{getDBConn}{Get the names the database server, user, and pass or returns error message}{getDBConn}
%
\begin{Description}\relax
Get the names the database server, user, and pass or returns error message
\end{Description}
%
\begin{Usage}
\begin{verbatim}
getDBConn()
\end{verbatim}
\end{Usage}
%
\begin{Value}
print the database connection information
\end{Value}
\inputencoding{utf8}
\HeaderA{hello}{Hello, World!}{hello}
%
\begin{Description}\relax
Prints 'Hello, world!'.
\end{Description}
%
\begin{Usage}
\begin{verbatim}
hello()
\end{verbatim}
\end{Usage}
%
\begin{Examples}
\begin{ExampleCode}
hello()
\end{ExampleCode}
\end{Examples}
\inputencoding{utf8}
\HeaderA{import.dictionary}{import the toxval and toxval\_type dictionaries}{import.dictionary}
%
\begin{Description}\relax
import the toxval and toxval\_type dictionaries
\end{Description}
%
\begin{Usage}
\begin{verbatim}
import.dictionary(toxval.db)
\end{verbatim}
\end{Usage}
%
\begin{Arguments}
\begin{ldescription}
\item[\code{toxval.db}] The name of the database
\end{ldescription}
\end{Arguments}
\inputencoding{utf8}
\HeaderA{import.source.info.by.source}{Load Source Info for each source into toxval The information is in the file \textasciitilde{}/dictionary/source\_info 2023-02-15.xlsx}{import.source.info.by.source}
%
\begin{Description}\relax
Load Source Info for each source into toxval
The information is in the file \textasciitilde{}/dictionary/source\_info 2023-02-15.xlsx
\end{Description}
%
\begin{Usage}
\begin{verbatim}
import.source.info.by.source(toxval.db, source = NULL)
\end{verbatim}
\end{Usage}
%
\begin{Arguments}
\begin{ldescription}
\item[\code{toxval.db}] The version of toxval into which the source info is loaded.

\item[\code{source}] The specific source to be loaded, If NULL, load for all sources
\end{ldescription}
\end{Arguments}
\inputencoding{utf8}
\HeaderA{load.dsstox}{Load DSSTox if needed from a file into a global variables (DSSTOX)}{load.dsstox}
%
\begin{Description}\relax
Load DSSTox if needed from a file into a global variables (DSSTOX)
\end{Description}
%
\begin{Usage}
\begin{verbatim}
load.dsstox()
\end{verbatim}
\end{Usage}
\inputencoding{utf8}
\HeaderA{log\_message}{Function to combine output log with output message}{log.Rul.message}
%
\begin{Description}\relax
Function to combine output log with output message

Function to combine output log with output message
\end{Description}
%
\begin{Usage}
\begin{verbatim}
log_message(log_df, message_df_col)

log_message(log_df, message_df_col)
\end{verbatim}
\end{Usage}
%
\begin{Arguments}
\begin{ldescription}
\item[\code{log\_df}] Dataframe to which the log information will be appended

\item[\code{message\_df\_col}] New message to add
\end{ldescription}
\end{Arguments}
\inputencoding{utf8}
\HeaderA{printCurrentFunction}{Print the name of the current function}{printCurrentFunction}
%
\begin{Description}\relax
Print the name of the current function
\end{Description}
%
\begin{Usage}
\begin{verbatim}
printCurrentFunction(comment.string = NA)
\end{verbatim}
\end{Usage}
%
\begin{Arguments}
\begin{ldescription}
\item[\code{comment.string}] An optional string to be printed
\end{ldescription}
\end{Arguments}
\inputencoding{utf8}
\HeaderA{runInsert}{Insert a record into a database. if auto.increment=TRUE, return the auto incremented primary key of the record. otherwise, return -1}{runInsert}
%
\begin{Description}\relax
Insert a record into a database. if auto.increment=TRUE, return the auto incremented
primary key of the record. otherwise, return -1
\end{Description}
%
\begin{Usage}
\begin{verbatim}
runInsert(query, db, do.halt = F, verbose = F, auto.increment.id = F)
\end{verbatim}
\end{Usage}
%
\begin{Arguments}
\begin{ldescription}
\item[\code{query}] a properly formatted SQL query as a string

\item[\code{db}] the name of the database

\item[\code{do.halt}] if TRUE, halt on errors or warnings

\item[\code{verbose}] if TRUE, print diagnostic information

\item[\code{auto.increment}] if TRUE, add the auto increment primary key even if not part of the query
\end{ldescription}
\end{Arguments}
%
\begin{Value}
Returns the database table auto incremented primary key ID
\end{Value}
\inputencoding{utf8}
\HeaderA{runInsertTable}{Inserts multiple rows into a database table}{runInsertTable}
%
\begin{Description}\relax
Inserts multiple rows into a database table
\end{Description}
%
\begin{Usage}
\begin{verbatim}
runInsertTable(mat, table, db, do.halt = T, verbose = F, get.id = T)
\end{verbatim}
\end{Usage}
%
\begin{Arguments}
\begin{ldescription}
\item[\code{mat}] data frame containing the data, with the column names corresponding

\item[\code{table}] name of the database table to which data will be inserted

\item[\code{db}] the name of the database

\item[\code{do.halt}] if TRUE, halt on errors or warnings

\item[\code{verbose}] if TRUE, print diagnostic information
\end{ldescription}
\end{Arguments}
\inputencoding{utf8}
\HeaderA{runQuery}{Runs a database query and returns a result set}{runQuery}
%
\begin{Description}\relax
Runs a database query and returns a result set
\end{Description}
%
\begin{Usage}
\begin{verbatim}
runQuery(query = NULL, db, do.halt = T, verbose = F)
\end{verbatim}
\end{Usage}
%
\begin{Arguments}
\begin{ldescription}
\item[\code{query}] a properly formatted SQL query as a string

\item[\code{db}] the name of the database

\item[\code{do.halt}] if TRUE, halt on errors or warnings

\item[\code{verbose}] if TRUE, print diagnostic information
\end{ldescription}
\end{Arguments}
\inputencoding{utf8}
\HeaderA{setDBConn}{set SQL connection to the database}{setDBConn}
%
\begin{Description}\relax
set SQL connection to the database
\end{Description}
%
\begin{Usage}
\begin{verbatim}
setDBConn(server = "ccte-mysql-res.epa.gov", user, password)
\end{verbatim}
\end{Usage}
%
\begin{Arguments}
\begin{ldescription}
\item[\code{server}] SQL server on which relevant database lives

\item[\code{user}] SQL username to access database

\item[\code{password}] SQL password corresponding to username
\end{ldescription}
\end{Arguments}
\inputencoding{utf8}
\HeaderA{source\_chemical.chemidplus}{special process to deal with source chemicals for ChemIDPlus}{source.Rul.chemical.chemidplus}
%
\begin{Description}\relax
special process to deal with source chemicals for ChemIDPlus
\end{Description}
%
\begin{Usage}
\begin{verbatim}
source_chemical.chemidplus(
  toxval.db,
  source.db,
  res,
  source,
  chem.check.halt = FALSE,
  casrn.col = "casrn",
  name.col = "name",
  verbose = F
)
\end{verbatim}
\end{Usage}
%
\begin{Arguments}
\begin{ldescription}
\item[\code{toxval.db}] The version of toxval into which the source info is loaded.

\item[\code{source.db}] The source database version

\item[\code{source}] The xource to be processed (ECOTOX)

\item[\code{chem.check.halt}] If TRUE, halt if there are errors in the chemical checking

\item[\code{casrn.col}] Name of the column containing the CASRN

\item[\code{name.col}] Name of the column containing chemical names

\item[\code{verbose}] If TRUE, output extra diagnostics information
\end{ldescription}
\end{Arguments}
\inputencoding{utf8}
\HeaderA{source\_chemical.ecotox}{special process to deal with source chemicals for ECOTOX}{source.Rul.chemical.ecotox}
%
\begin{Description}\relax
special process to deal with source chemicals for ECOTOX
\end{Description}
%
\begin{Usage}
\begin{verbatim}
source_chemical.ecotox(
  toxval.db,
  source.db,
  res,
  source,
  chem.check.halt = FALSE,
  casrn.col = "casrn",
  name.col = "name",
  verbose = F
)
\end{verbatim}
\end{Usage}
%
\begin{Arguments}
\begin{ldescription}
\item[\code{toxval.db}] The version of toxval into which the source info is loaded.

\item[\code{source.db}] The source database version

\item[\code{source}] The xource to be processed (ECOTOX)

\item[\code{chem.check.halt}] If TRUE, halt if there are errors in the chemical checking

\item[\code{casrn.col}] Name of the column containing the CASRN

\item[\code{name.col}] Name of the column containing chemical names

\item[\code{verbose}] If TRUE, output extra diagnostics information
\end{ldescription}
\end{Arguments}
\inputencoding{utf8}
\HeaderA{source\_chemical.extra}{special process to deal with source chemicals for extra source (cancer, genetox, skin\_eye, etc)}{source.Rul.chemical.extra}
%
\begin{Description}\relax
special process to deal with source chemicals for extra source (cancer, genetox, skin\_eye, etc)
\end{Description}
%
\begin{Usage}
\begin{verbatim}
source_chemical.extra(
  toxval.db,
  source.db,
  res,
  source,
  chem.check.halt = FALSE,
  casrn.col = "casrn",
  name.col = "name",
  verbose = F
)
\end{verbatim}
\end{Usage}
%
\begin{Arguments}
\begin{ldescription}
\item[\code{toxval.db}] The version of toxval into which the source info is loaded.

\item[\code{source.db}] The source database version

\item[\code{source}] The source to be processed \#' @param chem.check.halt If TRUE, halt if there are errors in the chemical checking

\item[\code{casrn.col}] Name of the column containing the CASRN

\item[\code{name.col}] Name of the column containing chemical names

\item[\code{verbose}] If TRUE, output extra diagnostics information
\end{ldescription}
\end{Arguments}
\inputencoding{utf8}
\HeaderA{source\_chemical.toxrefdb}{Special process to deal with source chemicals for ToxRefDB. This will put the chemicals into the source database source\_chemical table}{source.Rul.chemical.toxrefdb}
%
\begin{Description}\relax
Special process to deal with source chemicals for ToxRefDB. This will put the
chemicals into the source database source\_chemical table
\end{Description}
%
\begin{Usage}
\begin{verbatim}
source_chemical.toxrefdb(
  toxval.db,
  source.db,
  res,
  source = "ToxRefDB",
  chem.check.halt = FALSE,
  casrn.col = "casrn",
  name.col = "name",
  verbose = F
)
\end{verbatim}
\end{Usage}
%
\begin{Arguments}
\begin{ldescription}
\item[\code{toxval.db}] The version of toxval into which the source info is loaded.

\item[\code{source.db}] The source database version

\item[\code{res}] The dataframe to which the chemical\_id sill be added

\item[\code{source}] The name of the source

\item[\code{chem.check.halt}] If TRUE, stop if there are problems with the chemical mapping

\item[\code{casrn.col}] The name of the column containing the CASRN

\item[\code{name.col}] The name ofhte column containing hte chemical name

\item[\code{verbose}] If TRUE, write out diagnostic messages
\end{ldescription}
\end{Arguments}
%
\begin{Value}
Returns the input dataframe with the chemical\_id added
\end{Value}
\inputencoding{utf8}
\HeaderA{species.mapper}{Map the species to the ECOTOX species dictionary and export the missing species to add to the dictionary}{species.mapper}
%
\begin{Description}\relax
This function replaces fix.species
This function precedes toxvaldb.load.species
\end{Description}
%
\begin{Usage}
\begin{verbatim}
species.mapper(toxval.db, date_string = "2023-02-14")
\end{verbatim}
\end{Usage}
%
\begin{Arguments}
\begin{ldescription}
\item[\code{toxval.db}] The version of the database to use

\item[\code{date\_string}] The date of the dictionary versions
\end{ldescription}
\end{Arguments}
\inputencoding{utf8}
\HeaderA{toxval.config}{Define a set of global variables. These include the source path (datapath) and the source databases (e.g. dev\_toxval\_version and dev\_toxval\_source\_version) and the urls for the ACToR web services.}{toxval.config}
%
\begin{Description}\relax
Define a set of global variables. These include the source path (datapath)
and the source databases (e.g. dev\_toxval\_version and dev\_toxval\_source\_version)
and the urls for the ACToR web services.
\end{Description}
%
\begin{Usage}
\begin{verbatim}
toxval.config()
\end{verbatim}
\end{Usage}
%
\begin{Value}
Returns a set of parameters to be used throughout the package
\end{Value}
\inputencoding{utf8}
\HeaderA{toxval.init.db}{Initialize the database. THis sill load the species, info and dictionary tables}{toxval.init.db}
%
\begin{Description}\relax
Initialize the database. THis sill load the species, info and dictionary tables
\end{Description}
%
\begin{Usage}
\begin{verbatim}
toxval.init.db(toxval.db, reset = F, date_string = "2022-05-25")
\end{verbatim}
\end{Usage}
%
\begin{Arguments}
\begin{ldescription}
\item[\code{toxval.db}] The version of toxval into which the tables are loaded.

\item[\code{reset}] If TRUE, delete all content from the database

\item[\code{date\_string}] The date of the dictionary versions
\end{ldescription}
\end{Arguments}
\inputencoding{utf8}
\HeaderA{toxval.load.alaska\_dec}{Load the alaska\_dec (old ACToR - flex)data  from toxval sourcedb to toxval}{toxval.load.alaska.Rul.dec}
%
\begin{Description}\relax
Load the alaska\_dec (old ACToR - flex)data  from toxval sourcedb to toxval
\end{Description}
%
\begin{Usage}
\begin{verbatim}
toxval.load.alaska_dec(toxval.db, source.db, log = F)
\end{verbatim}
\end{Usage}
%
\begin{Arguments}
\begin{ldescription}
\item[\code{toxval.db}] The database version to use

\item[\code{source.db}] The source database

\item[\code{log}] If TRUE, output log inoformation to a file
\end{ldescription}
\end{Arguments}
\inputencoding{utf8}
\HeaderA{toxval.load.all}{Load and process all information into ToxValDB. The entire process can be run with one command: toxval.load.all(toxval.db=...,source.db=..., do.all=T) It can also be run in stages, but needs to be run in the order of the do.X parameters listed here. If any earlier step is run, all of the subsequent steps need to be rerun.}{toxval.load.all}
%
\begin{Description}\relax
Load and process all information into ToxValDB. The entire process can be run with
one command: toxval.load.all(toxval.db=...,source.db=..., do.all=T)
It can also be run in stages, but needs to be run in the order of the do.X parameters
listed here. If any earlier step is run, all of the subsequent steps need to be rerun.
\end{Description}
%
\begin{Usage}
\begin{verbatim}
toxval.load.all(
  toxval.db,
  source.db,
  log = F,
  do.init = F,
  do.reset = F,
  do.load = F,
  do.post = F,
  do.extra = F
)
\end{verbatim}
\end{Usage}
%
\begin{Arguments}
\begin{ldescription}
\item[\code{toxval.db}] The version of toxval into which the tables are loaded.

\item[\code{source.db}] The version of toxval\_source database from which information is pulled.

\item[\code{log}] If TRUE write the output from each load script to a log file

\item[\code{do.init}] If True, clean out all of the database tables

\item[\code{do.reset}] If TRUE, empty the database to restart

\item[\code{do.load}] If TRUE, load all of the source

\item[\code{do.post}] If TRUE, do th post-processingwork of fixing study type and risk\_assessment\_class

\item[\code{do.extra}] If TRUE, load the non-toxval data (genetox, bcfbaf, skin/eye)
\end{ldescription}
\end{Arguments}
%
\begin{Value}
Nothing is returned
\end{Value}
\inputencoding{utf8}
\HeaderA{toxval.load.atsdr}{Load the ATSDR MRLs 2020 data from toxval\_source to toxval}{toxval.load.atsdr}
%
\begin{Description}\relax
Load the ATSDR MRLs 2020 data from toxval\_source to toxval
\end{Description}
%
\begin{Usage}
\begin{verbatim}
toxval.load.atsdr(toxval.db, source.db, log = F)
\end{verbatim}
\end{Usage}
%
\begin{Arguments}
\begin{ldescription}
\item[\code{toxval.db}] The version of toxval into which the tables are loaded.

\item[\code{source.db}] The source database to use.

\item[\code{log}] If TRUE, send output to a log file
\end{ldescription}
\end{Arguments}
\inputencoding{utf8}
\HeaderA{toxval.load.atsdr.pfas}{Load the original ATSDR PFAS from toxval\_source to toxval}{toxval.load.atsdr.pfas}
%
\begin{Description}\relax
Load the original ATSDR PFAS from toxval\_source to toxval
\end{Description}
%
\begin{Usage}
\begin{verbatim}
toxval.load.atsdr.pfas(toxval.db, source.db, log = F)
\end{verbatim}
\end{Usage}
%
\begin{Arguments}
\begin{ldescription}
\item[\code{toxval.db}] The version of toxval into which the tables are loaded.

\item[\code{source.db}] The source database to use.

\item[\code{log}] If TRUE, send output to a log file
\end{ldescription}
\end{Arguments}
\inputencoding{utf8}
\HeaderA{toxval.load.atsdr.pfas.2021}{Load data ATSDR PFAS 2021 data from toxval\_source to toxval}{toxval.load.atsdr.pfas.2021}
%
\begin{Description}\relax
Load data ATSDR PFAS 2021 data from toxval\_source to toxval
\end{Description}
%
\begin{Usage}
\begin{verbatim}
toxval.load.atsdr.pfas.2021(toxval.db, source.db, log = F)
\end{verbatim}
\end{Usage}
%
\begin{Arguments}
\begin{ldescription}
\item[\code{toxval.db}] The version of toxval into which the tables are loaded.

\item[\code{source.db}] The source database to use.

\item[\code{log}] If TRUE, send messages to a log file
\end{ldescription}
\end{Arguments}
\inputencoding{utf8}
\HeaderA{toxval.load.atsdr\_mrl\_2020}{Load the ATSDR MRLs 2020 data from toxval\_source to toxval}{toxval.load.atsdr.Rul.mrl.Rul.2020}
%
\begin{Description}\relax
Load the ATSDR MRLs 2020 data from toxval\_source to toxval
\end{Description}
%
\begin{Usage}
\begin{verbatim}
toxval.load.atsdr_mrl_2020(toxval.db, source.db, log = F)
\end{verbatim}
\end{Usage}
%
\begin{Arguments}
\begin{ldescription}
\item[\code{toxval.db}] The version of toxval into which the tables are loaded.

\item[\code{source.db}] The source database to use.

\item[\code{log}] If TRUE, send output to a log file
\end{ldescription}
\end{Arguments}
\inputencoding{utf8}
\HeaderA{toxval.load.atsdr\_mrl\_2022}{Load the ATSDR MRLs 2022 data from toxval\_source to toxval}{toxval.load.atsdr.Rul.mrl.Rul.2022}
%
\begin{Description}\relax
Load the ATSDR MRLs 2022 data from toxval\_source to toxval
\end{Description}
%
\begin{Usage}
\begin{verbatim}
toxval.load.atsdr_mrl_2022(toxval.db, source.db, log = F)
\end{verbatim}
\end{Usage}
%
\begin{Arguments}
\begin{ldescription}
\item[\code{toxval.db}] The version of toxval into which the tables are loaded.

\item[\code{source.db}] The source database to use.

\item[\code{log}] If TRUE, send output to a log file
\end{ldescription}
\end{Arguments}
\inputencoding{utf8}
\HeaderA{toxval.load.bcfbaf}{Load the Arnot BAF / BCF data}{toxval.load.bcfbaf}
%
\begin{Description}\relax
Load the Arnot BAF / BCF data
\end{Description}
%
\begin{Usage}
\begin{verbatim}
toxval.load.bcfbaf(toxval.db, source.db, verbose = F)
\end{verbatim}
\end{Usage}
%
\begin{Arguments}
\begin{ldescription}
\item[\code{toxval.db}] The database to use.

\item[\code{verbose}] If TRUE, print out extra diagnostic messages
\end{ldescription}
\end{Arguments}
\inputencoding{utf8}
\HeaderA{toxval.load.caloehha}{Load new\_caloehha from toxval\_source to toxval}{toxval.load.caloehha}
%
\begin{Description}\relax
Load new\_caloehha from toxval\_source to toxval
\end{Description}
%
\begin{Usage}
\begin{verbatim}
toxval.load.caloehha(toxval.db, source.db, log = F)
\end{verbatim}
\end{Usage}
%
\begin{Arguments}
\begin{ldescription}
\item[\code{toxval.db}] The version of toxval into which the tables are loaded.

\item[\code{source.db}] The source database to use.

\item[\code{log}] If TRUE, send output to a log file
\end{ldescription}
\end{Arguments}
\inputencoding{utf8}
\HeaderA{toxval.load.cal\_dph}{Load the California DPH data (old ACToR - flex)data  from toxval sourcedb to toxval}{toxval.load.cal.Rul.dph}
%
\begin{Description}\relax
Load the California DPH data (old ACToR - flex)data  from toxval sourcedb to toxval
\end{Description}
%
\begin{Usage}
\begin{verbatim}
toxval.load.cal_dph(toxval.db, source.db, log = F)
\end{verbatim}
\end{Usage}
%
\begin{Arguments}
\begin{ldescription}
\item[\code{toxval.db}] The database version to use

\item[\code{source.db}] The source database

\item[\code{log}] If TRUE, send output to a log file
\end{ldescription}
\end{Arguments}
\inputencoding{utf8}
\HeaderA{toxval.load.cancer}{prepare the cancer call data. The data comes form a series of files ../NIOSH/NIOSH\_CARC\_2018.xlsx ../IRIS/iris\_cancer\_call\_2018-10-03.xlsx ../PPRTV\_ORNL/PPRTV\_ORNL cancer calls 2018-10-25.xlsx ../cancer\_summary/cancer/NTP/NTP cancer clean.xlsx ../cancer\_summary/cancer/IARC/IARC cancer 2018-10-29.xlsx ../cancer\_summary/cancer/HealthCanada/HealthCanada\_TRVs\_2010\_AppendixA v2.xlsx ../cancer\_summary/cancer/EPA\_OPP\_CARC/EPA\_CARC.xlsx ../cancer\_summary/cancer/CalEPA/calepa\_p65\_cancer\_only.xlsx}{toxval.load.cancer}
%
\begin{Description}\relax
extract all of the chemicals with cancer slope factor or unit risk with appropriate units
\end{Description}
%
\begin{Usage}
\begin{verbatim}
toxval.load.cancer(toxval.db, source.db)
\end{verbatim}
\end{Usage}
%
\begin{Arguments}
\begin{ldescription}
\item[\code{toxval.db}] The version of the database to use
\end{ldescription}
\end{Arguments}
\inputencoding{utf8}
\HeaderA{toxval.load.chemidplus}{Load ChemID Plus Acute data data to toxval}{toxval.load.chemidplus}
%
\begin{Description}\relax
Load ChemID Plus Acute data data to toxval
\end{Description}
%
\begin{Usage}
\begin{verbatim}
toxval.load.chemidplus(toxval.db, source.db, log = F, do.init = F)
\end{verbatim}
\end{Usage}
%
\begin{Arguments}
\begin{ldescription}
\item[\code{toxval.db}] The version of toxval into which the tables are loaded.

\item[\code{log}] If TRUE, send output to a log file

\item[\code{do.init}] if TRUE, read the data in from the file and set up the matrix

\item[\code{verbose}] Whether the loaded rows should be printed to the console.
\end{ldescription}
\end{Arguments}
\inputencoding{utf8}
\HeaderA{toxval.load.chiu}{Load the Chiu data from toxval\_source to toxval}{toxval.load.chiu}
%
\begin{Description}\relax
Load the Chiu data from toxval\_source to toxval
\end{Description}
%
\begin{Usage}
\begin{verbatim}
toxval.load.chiu(toxval.db, source.db, log = F)
\end{verbatim}
\end{Usage}
%
\begin{Arguments}
\begin{ldescription}
\item[\code{toxval.db}] The version of toxval into which the tables are loaded.

\item[\code{source.db}] The source database to use.

\item[\code{log}] If TRUE, send output to a log file
\end{ldescription}
\end{Arguments}
\inputencoding{utf8}
\HeaderA{toxval.load.copper}{Load Copper Manufacturers daa from toxval\_source to toxval}{toxval.load.copper}
%
\begin{Description}\relax
Load Copper Manufacturers daa from toxval\_source to toxval
\end{Description}
%
\begin{Usage}
\begin{verbatim}
toxval.load.copper(toxval.db, source.db, log = F)
\end{verbatim}
\end{Usage}
%
\begin{Arguments}
\begin{ldescription}
\item[\code{toxval.db}] The version of toxval into which the tables are loaded.

\item[\code{source.db}] The source database to use.

\item[\code{log}] If TRUE, send output to a log file
\end{ldescription}
\end{Arguments}
\inputencoding{utf8}
\HeaderA{toxval.load.cosmos}{Load teh COSMOS data from source to toxval}{toxval.load.cosmos}
%
\begin{Description}\relax
Load teh COSMOS data from source to toxval
\end{Description}
%
\begin{Usage}
\begin{verbatim}
toxval.load.cosmos(toxval.db, source.db, log = F)
\end{verbatim}
\end{Usage}
%
\begin{Arguments}
\begin{ldescription}
\item[\code{toxval.db}] The version of toxval into which the tables are loaded.

\item[\code{source.db}] The source database to use.

\item[\code{log}] If TRUE, send output to a log file
\end{ldescription}
\end{Arguments}
\inputencoding{utf8}
\HeaderA{toxval.load.dod}{Load the DOD data from toxval\_source to toxval}{toxval.load.dod}
%
\begin{Description}\relax
Load the DOD data from toxval\_source to toxval
\end{Description}
%
\begin{Usage}
\begin{verbatim}
toxval.load.dod(toxval.db, source.db, log = F)
\end{verbatim}
\end{Usage}
%
\begin{Arguments}
\begin{ldescription}
\item[\code{toxval.db}] The version of toxval into which the tables are loaded.

\item[\code{source.db}] The source database to use.

\item[\code{log}] If TRUE, send output to a log file
\end{ldescription}
\end{Arguments}
\inputencoding{utf8}
\HeaderA{toxval.load.dod.ered}{Load the DOD ERED data from toxval\_source to toxval}{toxval.load.dod.ered}
%
\begin{Description}\relax
Load the DOD ERED data from toxval\_source to toxval
\end{Description}
%
\begin{Usage}
\begin{verbatim}
toxval.load.dod.ered(toxval.db, source.db, log = F)
\end{verbatim}
\end{Usage}
%
\begin{Arguments}
\begin{ldescription}
\item[\code{toxval.db}] The version of toxval into which the tables are loaded.

\item[\code{source.db}] The source database to use.

\item[\code{log}] If TRUE, send output to a log file
\end{ldescription}
\end{Arguments}
\inputencoding{utf8}
\HeaderA{toxval.load.doe.benchmarks}{Load DOE Wildlife Benchmarks data from toxval\_source to toxval}{toxval.load.doe.benchmarks}
%
\begin{Description}\relax
Load DOE Wildlife Benchmarks data from toxval\_source to toxval
\end{Description}
%
\begin{Usage}
\begin{verbatim}
toxval.load.doe.benchmarks(toxval.db, source.db, log = F)
\end{verbatim}
\end{Usage}
%
\begin{Arguments}
\begin{ldescription}
\item[\code{toxval.db}] The version of toxval into which the tables are loaded.

\item[\code{source.db}] The source database to use.

\item[\code{log}] If TRUE, send output to a log file
\end{ldescription}
\end{Arguments}
\inputencoding{utf8}
\HeaderA{toxval.load.doe.ecorisk}{Load the DOE ECORISK data (also called LANL) data from toxval\_source to toxval}{toxval.load.doe.ecorisk}
%
\begin{Description}\relax
Load the DOE ECORISK data (also called LANL) data from toxval\_source to toxval
\end{Description}
%
\begin{Usage}
\begin{verbatim}
toxval.load.doe.ecorisk(toxval.db, source.db, log = F)
\end{verbatim}
\end{Usage}
%
\begin{Arguments}
\begin{ldescription}
\item[\code{toxval.db}] The version of toxval into which the tables are loaded.

\item[\code{source.db}] The version of toxval\_source from which the tables are loaded.

\item[\code{log}] If TRUE, send output to a log file
\end{ldescription}
\end{Arguments}
\inputencoding{utf8}
\HeaderA{toxval.load.doe.pac}{Load DOE Protective Action Criteria data from toxval\_source to toxval}{toxval.load.doe.pac}
%
\begin{Description}\relax
Load DOE Protective Action Criteria data from toxval\_source to toxval
\end{Description}
%
\begin{Usage}
\begin{verbatim}
toxval.load.doe.pac(toxval.db, source.db, log = F)
\end{verbatim}
\end{Usage}
%
\begin{Arguments}
\begin{ldescription}
\item[\code{toxval.db}] The version of toxval into which the tables are loaded.

\item[\code{source.db}] The source database to use.

\item[\code{log}] If TRUE, send output to a log file
\end{ldescription}
\end{Arguments}
\inputencoding{utf8}
\HeaderA{toxval.load.echa.echemportal.api}{Load ECHA eChemPortal API data from toxval\_source to toxval}{toxval.load.echa.echemportal.api}
%
\begin{Description}\relax
Load ECHA eChemPortal API data from toxval\_source to toxval
\end{Description}
%
\begin{Usage}
\begin{verbatim}
toxval.load.echa.echemportal.api(toxval.db, source.db, log = F)
\end{verbatim}
\end{Usage}
%
\begin{Arguments}
\begin{ldescription}
\item[\code{toxval.db}] The version of toxval into which the tables are loaded.

\item[\code{source.db}] The source database to use.

\item[\code{log}] If TRUE, send output to a log file
\end{ldescription}
\end{Arguments}
\inputencoding{utf8}
\HeaderA{toxval.load.echa\_iuclid}{Loading the ECHA IUCLID data to toxval from toxval\_source This method is different from most because there are multiple tables (one per study type) for this source}{toxval.load.echa.Rul.iuclid}
%
\begin{Description}\relax
Loading the ECHA IUCLID data to toxval from toxval\_source
This method is different from most because there are multiple tables (one per study
type) for this source
\end{Description}
%
\begin{Usage}
\begin{verbatim}
toxval.load.echa_iuclid(toxval.db, source.db, log = F, reset = F)
\end{verbatim}
\end{Usage}
%
\begin{Arguments}
\begin{ldescription}
\item[\code{toxval.db}] The database version to use

\item[\code{source.db}] The source database

\item[\code{log}] If TRUE, send output to a log file
\end{ldescription}
\end{Arguments}
\inputencoding{utf8}
\HeaderA{toxval.load.ecotox}{Load ECOTOX from the datahub to toxval}{toxval.load.ecotox}
%
\begin{Description}\relax
Load ECOTOX from the datahub to toxval
\end{Description}
%
\begin{Usage}
\begin{verbatim}
toxval.load.ecotox(
  toxval.db,
  source.db,
  log = F,
  do.load = F,
  sys.date = "2023-01-26"
)
\end{verbatim}
\end{Usage}
%
\begin{Arguments}
\begin{ldescription}
\item[\code{toxval.db}] The version of toxval into which the tables are loaded.

\item[\code{source.db}] The version of toxval source - used to manage chemicals

\item[\code{log}] If TRUE, send output to a log file

\item[\code{do.load}] If TRUE, load the data from the input file and put into a global variable

\item[\code{verbose}] Whether the loaded rows should be printed to the console.
\end{ldescription}
\end{Arguments}
\inputencoding{utf8}
\HeaderA{toxval.load.efsa}{Load EFSA data from toxval\_source to toxval}{toxval.load.efsa}
%
\begin{Description}\relax
Load EFSA data from toxval\_source to toxval
\end{Description}
%
\begin{Usage}
\begin{verbatim}
toxval.load.efsa(toxval.db, source.db, log = F)
\end{verbatim}
\end{Usage}
%
\begin{Arguments}
\begin{ldescription}
\item[\code{toxval.db}] The version of toxval into which the tables are loaded.

\item[\code{source.db}] The source database to use.

\item[\code{log}] If TRUE, send output to a log file
\end{ldescription}
\end{Arguments}
\inputencoding{utf8}
\HeaderA{toxval.load.efsa2}{Load EFSA2 data from toxval\_source to toxval}{toxval.load.efsa2}
%
\begin{Description}\relax
Load EFSA2 data from toxval\_source to toxval
\end{Description}
%
\begin{Usage}
\begin{verbatim}
toxval.load.efsa2(toxval.db, source.db, log = F)
\end{verbatim}
\end{Usage}
%
\begin{Arguments}
\begin{ldescription}
\item[\code{toxval.db}] The version of toxval into which the tables are loaded.

\item[\code{source.db}] The source databse from which data should be loaded

\item[\code{log}] If TRUE, send output to a log file
\end{ldescription}
\end{Arguments}
\inputencoding{utf8}
\HeaderA{toxval.load.envirotox}{Load EnviroTox data from toxval\_source to toxval}{toxval.load.envirotox}
%
\begin{Description}\relax
Load EnviroTox data from toxval\_source to toxval
\end{Description}
%
\begin{Usage}
\begin{verbatim}
toxval.load.envirotox(toxval.db, source.db, log = F)
\end{verbatim}
\end{Usage}
%
\begin{Arguments}
\begin{ldescription}
\item[\code{toxval.db}] The version of toxval into which the tables are loaded.

\item[\code{source.db}] The source database to use.

\item[\code{log}] If TRUE, send output to a log file
\end{ldescription}
\end{Arguments}
\inputencoding{utf8}
\HeaderA{toxval.load.epa\_aegl}{Load the EPA AEGL (old ACToR - flex)data  from toxval sourcedb to toxval}{toxval.load.epa.Rul.aegl}
%
\begin{Description}\relax
Load the EPA AEGL (old ACToR - flex)data  from toxval sourcedb to toxval
\end{Description}
%
\begin{Usage}
\begin{verbatim}
toxval.load.epa_aegl(toxval.db, source.db, log = F)
\end{verbatim}
\end{Usage}
%
\begin{Arguments}
\begin{ldescription}
\item[\code{toxval.db}] The database version to use

\item[\code{source.db}] The source database

\item[\code{log}] If TRUE, send output to a log file
\end{ldescription}
\end{Arguments}
\inputencoding{utf8}
\HeaderA{toxval.load.fda\_cedi}{Load the FDA CEDI (old ACToR - flex)data  from toxval sourcedb to toxval}{toxval.load.fda.Rul.cedi}
%
\begin{Description}\relax
Load the FDA CEDI (old ACToR - flex)data  from toxval sourcedb to toxval
\end{Description}
%
\begin{Usage}
\begin{verbatim}
toxval.load.fda_cedi(toxval.db, source.db, log = F)
\end{verbatim}
\end{Usage}
%
\begin{Arguments}
\begin{ldescription}
\item[\code{toxval.db}] The database version to use

\item[\code{source.db}] The source database

\item[\code{log}] If TRUE, send output to a log file
\end{ldescription}
\end{Arguments}
\inputencoding{utf8}
\HeaderA{toxval.load.flex}{Load the FLEX data (old ACToR data) from files to toxval. This will load all Excel file in the folder ACToR replacements/}{toxval.load.flex}
%
\begin{Description}\relax
Load the FLEX data (old ACToR data) from files to toxval. This will load all
Excel file in the folder ACToR replacements/
\end{Description}
%
\begin{Usage}
\begin{verbatim}
toxval.load.flex(toxval.db, verbose = F, only.new = F)
\end{verbatim}
\end{Usage}
%
\begin{Arguments}
\begin{ldescription}
\item[\code{toxval.db}] The version of toxval into which the tables are loaded.

\item[\code{verbose}] Whether the loaded rows should be printed to the console.

\item[\code{only.new}] if TRUE only files where the sources is not alrady in the database
will be loaded
\end{ldescription}
\end{Arguments}
\inputencoding{utf8}
\HeaderA{toxval.load.generic}{Generic structure for laoding to toxval from toxval\_source}{toxval.load.generic}
%
\begin{Description}\relax
Generic structure for laoding to toxval from toxval\_source
\end{Description}
%
\begin{Usage}
\begin{verbatim}
toxval.load.generic(toxvaldb, source.db, log = F)
\end{verbatim}
\end{Usage}
%
\begin{Arguments}
\begin{ldescription}
\item[\code{source.db}] The source database

\item[\code{log}] If TRUE, send output to a log file

\item[\code{toxval.db}] The database version to use
\end{ldescription}
\end{Arguments}
\inputencoding{utf8}
\HeaderA{toxval.load.genetox.all}{Load the Genetox data from Grace}{toxval.load.genetox.all}
%
\begin{Description}\relax
Load the Genetox data from Grace
\end{Description}
%
\begin{Usage}
\begin{verbatim}
toxval.load.genetox.all(
  toxval.db,
  source.db,
  sys.date = "2021-09-10",
  verbose = FALSE
)
\end{verbatim}
\end{Usage}
%
\begin{Arguments}
\begin{ldescription}
\item[\code{toxval.db}] The database to use.

\item[\code{verbose}] If TRUE output debug information

\item[\code{do.read}] If TRUE, read in the DSSTox file
\end{ldescription}
\end{Arguments}
\inputencoding{utf8}
\HeaderA{toxval.load.hawc}{Load HAWC from toxval\_source to toxval}{toxval.load.hawc}
%
\begin{Description}\relax
Load HAWC from toxval\_source to toxval
\end{Description}
%
\begin{Usage}
\begin{verbatim}
toxval.load.hawc(toxval.db, source.db, log = F)
\end{verbatim}
\end{Usage}
%
\begin{Arguments}
\begin{ldescription}
\item[\code{toxval.db}] The version of toxval into which the tables are loaded.

\item[\code{source.db}] The version of toxval\_source from which the tables are loaded.

\item[\code{log}] If TRUE, send output to a log file
\end{ldescription}
\end{Arguments}
\inputencoding{utf8}
\HeaderA{toxval.load.hawc\_pfas\_150}{Load HAWC PFAS 150 from toxval\_source to toxval}{toxval.load.hawc.Rul.pfas.Rul.150}
%
\begin{Description}\relax
Load HAWC PFAS 150 from toxval\_source to toxval
\end{Description}
%
\begin{Usage}
\begin{verbatim}
toxval.load.hawc_pfas_150(toxval.db, source.db, log = F)
\end{verbatim}
\end{Usage}
%
\begin{Arguments}
\begin{ldescription}
\item[\code{toxval.db}] The version of toxval into which the tables are loaded.

\item[\code{source.db}] The version of toxval\_source from which the tables are loaded.

\item[\code{log}] If TRUE, send output to a log file
\end{ldescription}
\end{Arguments}
\inputencoding{utf8}
\HeaderA{toxval.load.hawc\_pfas\_430}{Load HAWC PFAS 430 from toxval\_source to toxval}{toxval.load.hawc.Rul.pfas.Rul.430}
%
\begin{Description}\relax
Load HAWC PFAS 430 from toxval\_source to toxval
\end{Description}
%
\begin{Usage}
\begin{verbatim}
toxval.load.hawc_pfas_430(toxval.db, source.db, log = F)
\end{verbatim}
\end{Usage}
%
\begin{Arguments}
\begin{ldescription}
\item[\code{toxval.db}] The version of toxval into which the tables are loaded.

\item[\code{source.db}] The version of toxval\_source from which the tables are loaded.

\item[\code{log}] If TRUE, send output to a log file
\end{ldescription}
\end{Arguments}
\inputencoding{utf8}
\HeaderA{toxval.load.healthcanada}{Load Health Canada data from toxval\_source to toxval}{toxval.load.healthcanada}
%
\begin{Description}\relax
Load Health Canada data from toxval\_source to toxval
\end{Description}
%
\begin{Usage}
\begin{verbatim}
toxval.load.healthcanada(toxval.db, source.db, log = F)
\end{verbatim}
\end{Usage}
%
\begin{Arguments}
\begin{ldescription}
\item[\code{toxval.db}] The version of toxval into which the tables are loaded.

\item[\code{source.db}] The version of toxval\_source from which the tables are loaded.

\item[\code{log}] If TRUE, send output to a log file
\end{ldescription}
\end{Arguments}
\inputencoding{utf8}
\HeaderA{toxval.load.heast}{Load the HEAST data from toxval\_source to toxval}{toxval.load.heast}
%
\begin{Description}\relax
Load the HEAST data from toxval\_source to toxval
\end{Description}
%
\begin{Usage}
\begin{verbatim}
toxval.load.heast(toxval.db, source.db, log = F)
\end{verbatim}
\end{Usage}
%
\begin{Arguments}
\begin{ldescription}
\item[\code{toxval.db}] The version of toxval into which the tables are loaded.

\item[\code{source.db}] The source database to use.

\item[\code{log}] If TRUE, send output to a log file
\end{ldescription}
\end{Arguments}
\inputencoding{utf8}
\HeaderA{toxval.load.hess}{Load the HESS data from toxval\_source to toxval}{toxval.load.hess}
%
\begin{Description}\relax
Load the HESS data from toxval\_source to toxval
\end{Description}
%
\begin{Usage}
\begin{verbatim}
toxval.load.hess(toxval.db, source.db, log = F)
\end{verbatim}
\end{Usage}
%
\begin{Arguments}
\begin{ldescription}
\item[\code{toxval.db}] The version of toxval into which the tables are loaded.

\item[\code{source.db}] The source database to use.

\item[\code{log}] If TRUE, send output to a log file
\end{ldescription}
\end{Arguments}
\inputencoding{utf8}
\HeaderA{toxval.load.hpvis}{Load HPVIS from toxval\_source to toxval}{toxval.load.hpvis}
%
\begin{Description}\relax
Load HPVIS from toxval\_source to toxval
\end{Description}
%
\begin{Usage}
\begin{verbatim}
toxval.load.hpvis(toxval.db, source.db, log = F)
\end{verbatim}
\end{Usage}
%
\begin{Arguments}
\begin{ldescription}
\item[\code{toxval.db}] The version of toxval into which the tables are loaded.

\item[\code{source.db}] The source databse from which data should be loaded

\item[\code{log}] If TRUE, send output to a log file
\end{ldescription}
\end{Arguments}
\inputencoding{utf8}
\HeaderA{toxval.load.initial}{Delete the contents of the toxval database}{toxval.load.initial}
%
\begin{Description}\relax
Delete the contents of the toxval database
\end{Description}
%
\begin{Usage}
\begin{verbatim}
toxval.load.initial(toxval.db)
\end{verbatim}
\end{Usage}
%
\begin{Arguments}
\begin{ldescription}
\item[\code{toxval.db}] The version of toxval from which the data is deleted.
\end{ldescription}
\end{Arguments}
\inputencoding{utf8}
\HeaderA{toxval.load.iris}{Load new\_iris\_noncancer and new\_iris\_cancer from toxval\_source to toxval}{toxval.load.iris}
%
\begin{Description}\relax
Load new\_iris\_noncancer and new\_iris\_cancer from toxval\_source to toxval
\end{Description}
%
\begin{Usage}
\begin{verbatim}
toxval.load.iris(toxval.db, source.db, log = F)
\end{verbatim}
\end{Usage}
%
\begin{Arguments}
\begin{ldescription}
\item[\code{toxval.db}] The version of toxval into which the tables are loaded.

\item[\code{source.db}] The source database to use.

\item[\code{log}] If TRUE, send output to a log file
\end{ldescription}
\end{Arguments}
\inputencoding{utf8}
\HeaderA{toxval.load.mass\_mmcl}{Load the mass\_mmcl (old ACToR - flex)data  from toxval sourcedb to toxval}{toxval.load.mass.Rul.mmcl}
%
\begin{Description}\relax
Load the mass\_mmcl (old ACToR - flex)data  from toxval sourcedb to toxval
\end{Description}
%
\begin{Usage}
\begin{verbatim}
toxval.load.mass_mmcl(toxval.db, source.db, log = F)
\end{verbatim}
\end{Usage}
%
\begin{Arguments}
\begin{ldescription}
\item[\code{toxval.db}] The database version to use

\item[\code{source.db}] The source database

\item[\code{log}] If TRUE, send output to a log file
\end{ldescription}
\end{Arguments}
\inputencoding{utf8}
\HeaderA{toxval.load.niosh}{Load NIOSH from toxval\_source to toxval}{toxval.load.niosh}
%
\begin{Description}\relax
Load NIOSH from toxval\_source to toxval
\end{Description}
%
\begin{Usage}
\begin{verbatim}
toxval.load.niosh(toxval.db, source.db, log = F)
\end{verbatim}
\end{Usage}
%
\begin{Arguments}
\begin{ldescription}
\item[\code{toxval.db}] The version of toxval into which the tables are loaded.

\item[\code{source.db}] The source database to use.

\item[\code{log}] If TRUE, send output to a log file
\end{ldescription}
\end{Arguments}
\inputencoding{utf8}
\HeaderA{toxval.load.opp}{Load opp from toxval\_source to toxval}{toxval.load.opp}
%
\begin{Description}\relax
Load opp from toxval\_source to toxval
\end{Description}
%
\begin{Usage}
\begin{verbatim}
toxval.load.opp(toxval.db, source.db, log = F)
\end{verbatim}
\end{Usage}
%
\begin{Arguments}
\begin{ldescription}
\item[\code{toxval.db}] The version of toxval into which the tables are loaded.

\item[\code{source.db}] The version of toxval\_source from which the tables are loaded.

\item[\code{log}] If TRUE, send output to a log file
\end{ldescription}
\end{Arguments}
\inputencoding{utf8}
\HeaderA{toxval.load.oppt}{Load new\_oppt\_table from toxval\_source to toxval}{toxval.load.oppt}
%
\begin{Description}\relax
Load new\_oppt\_table from toxval\_source to toxval
\end{Description}
%
\begin{Usage}
\begin{verbatim}
toxval.load.oppt(toxval.db, source.db, log = F)
\end{verbatim}
\end{Usage}
%
\begin{Arguments}
\begin{ldescription}
\item[\code{toxval.db}] The version of toxval into which the tables are loaded.

\item[\code{source.db}] The source database to use.

\item[\code{log}] If TRUE, send output to a log file
\end{ldescription}
\end{Arguments}
\inputencoding{utf8}
\HeaderA{toxval.load.osha\_air\_limits}{Load the osha\_air\_limits (old ACToR - flex)data  from toxval sourcedb to toxval}{toxval.load.osha.Rul.air.Rul.limits}
%
\begin{Description}\relax
Load the osha\_air\_limits (old ACToR - flex)data  from toxval sourcedb to toxval
\end{Description}
%
\begin{Usage}
\begin{verbatim}
toxval.load.osha_air_limits(toxval.db, source.db, log = F)
\end{verbatim}
\end{Usage}
%
\begin{Arguments}
\begin{ldescription}
\item[\code{toxval.db}] The database version to use

\item[\code{source.db}] The source database

\item[\code{log}] If TRUE, send output to a log file
\end{ldescription}
\end{Arguments}
\inputencoding{utf8}
\HeaderA{toxval.load.ow\_dwsha}{Load the ow\_dwsha (old ACToR - flex) data  from toxval sourcedb to toxval}{toxval.load.ow.Rul.dwsha}
%
\begin{Description}\relax
Load the ow\_dwsha (old ACToR - flex) data  from toxval sourcedb to toxval
\end{Description}
%
\begin{Usage}
\begin{verbatim}
toxval.load.ow_dwsha(toxval.db, source.db, log = F)
\end{verbatim}
\end{Usage}
%
\begin{Arguments}
\begin{ldescription}
\item[\code{toxval.db}] The database version to use

\item[\code{source.db}] The source database

\item[\code{log}] If TRUE, send output to a log file
\end{ldescription}
\end{Arguments}
\inputencoding{utf8}
\HeaderA{toxval.load.penn}{Load Penn data from toxval\_source to toxval}{toxval.load.penn}
%
\begin{Description}\relax
Load Penn data from toxval\_source to toxval
\end{Description}
%
\begin{Usage}
\begin{verbatim}
toxval.load.penn(toxval.db, source.db, log = F)
\end{verbatim}
\end{Usage}
%
\begin{Arguments}
\begin{ldescription}
\item[\code{toxval.db}] The version of toxval into which the tables are loaded.

\item[\code{source.db}] The source database to use.

\item[\code{log}] If TRUE, send output to a log file
\end{ldescription}
\end{Arguments}
\inputencoding{utf8}
\HeaderA{toxval.load.penn\_dep}{Load the penn\_dep (old ACToR - flex)data  from toxval sourcedb to toxval}{toxval.load.penn.Rul.dep}
%
\begin{Description}\relax
Load the penn\_dep (old ACToR - flex)data  from toxval sourcedb to toxval
\end{Description}
%
\begin{Usage}
\begin{verbatim}
toxval.load.penn_dep(toxval.db, source.db, log = F)
\end{verbatim}
\end{Usage}
%
\begin{Arguments}
\begin{ldescription}
\item[\code{toxval.db}] The database version to use

\item[\code{source.db}] The source database

\item[\code{log}] If TRUE, send output to a log file
\end{ldescription}
\end{Arguments}
\inputencoding{utf8}
\HeaderA{toxval.load.pfas\_150\_sem\_v2}{Load pfas\_150\_sem from toxval\_source to toxval}{toxval.load.pfas.Rul.150.Rul.sem.Rul.v2}
%
\begin{Description}\relax
Load pfas\_150\_sem from toxval\_source to toxval
\end{Description}
%
\begin{Usage}
\begin{verbatim}
toxval.load.pfas_150_sem_v2(toxval.db, source.db, log = F)
\end{verbatim}
\end{Usage}
%
\begin{Arguments}
\begin{ldescription}
\item[\code{toxval.db}] The version of toxval into which the tables are loaded.

\item[\code{source.db}] The source database to use.

\item[\code{log}] If TRUE, send output to a log file
\end{ldescription}
\end{Arguments}
\inputencoding{utf8}
\HeaderA{toxval.load.postprocess}{Do all of the post-processing steps for a source}{toxval.load.postprocess}
%
\begin{Description}\relax
Do all of the post-processing steps for a source
\end{Description}
%
\begin{Usage}
\begin{verbatim}
toxval.load.postprocess(
  toxval.db,
  source.db,
  source,
  do.convert.units = F,
  chem_source,
  subsource = NULL
)
\end{verbatim}
\end{Usage}
%
\begin{Arguments}
\begin{ldescription}
\item[\code{toxval.db}] The database version to use

\item[\code{source}] The source name

\item[\code{do.convert.units}] If TRUE, convert units, mainly from ppm to mg/kg-day. This code is not debugged

\item[\code{chem\_source}] Used only for source=ECHA IUCLID

\item[\code{sourcedb}] The source database name
\end{ldescription}
\end{Arguments}
\inputencoding{utf8}
\HeaderA{toxval.load.pprtv.cphea}{Load PPRTV (CPHEA) from toxval source to toxval}{toxval.load.pprtv.cphea}
%
\begin{Description}\relax
Load PPRTV (CPHEA) from toxval source to toxval
\end{Description}
%
\begin{Usage}
\begin{verbatim}
toxval.load.pprtv.cphea(toxval.db, source.db, log = F)
\end{verbatim}
\end{Usage}
%
\begin{Arguments}
\begin{ldescription}
\item[\code{toxval.db}] The version of toxval into which the tables are loaded.

\item[\code{source.db}] The version of toxval\_source from which the tables are loaded.

\item[\code{log}] If TRUE, send output to a log file
\end{ldescription}
\end{Arguments}
\inputencoding{utf8}
\HeaderA{toxval.load.pprtv.ncea}{Load PPRTV (NCEA) from toxval source to toxval}{toxval.load.pprtv.ncea}
%
\begin{Description}\relax
Load PPRTV (NCEA) from toxval source to toxval
\end{Description}
%
\begin{Usage}
\begin{verbatim}
toxval.load.pprtv.ncea(toxval.db, source.db, log = F)
\end{verbatim}
\end{Usage}
%
\begin{Arguments}
\begin{ldescription}
\item[\code{toxval.db}] The version of toxval into which the tables are loaded.

\item[\code{source.db}] The version of toxval\_source from which the tables are loaded.

\item[\code{log}] If TRUE, send output to a log file
\end{ldescription}
\end{Arguments}
\inputencoding{utf8}
\HeaderA{toxval.load.pprtv.ornl}{Load PPRTV (ORNL) from toxval\_source to toxval}{toxval.load.pprtv.ornl}
%
\begin{Description}\relax
Load PPRTV (ORNL) from toxval\_source to toxval
\end{Description}
%
\begin{Usage}
\begin{verbatim}
toxval.load.pprtv.ornl(toxval.db, source.db, log = F)
\end{verbatim}
\end{Usage}
%
\begin{Arguments}
\begin{ldescription}
\item[\code{toxval.db}] The version of toxval into which the tables are loaded.

\item[\code{source.db}] The source databse from which data should be loaded

\item[\code{log}] If TRUE, send output to a log file
\end{ldescription}
\end{Arguments}
\inputencoding{utf8}
\HeaderA{toxval.load.rsl}{Load the RSL data from source db to toxval - the source database needs to be updated periodically}{toxval.load.rsl}
%
\begin{Description}\relax
Load the RSL data from source db to toxval - the source database needs to be updated periodically
\end{Description}
%
\begin{Usage}
\begin{verbatim}
toxval.load.rsl(toxval.db, source.db, log = F)
\end{verbatim}
\end{Usage}
%
\begin{Arguments}
\begin{ldescription}
\item[\code{toxval.db}] The database version to use

\item[\code{source.db}] The source database

\item[\code{log}] If TRUE, send output to a log file
\end{ldescription}
\end{Arguments}
\inputencoding{utf8}
\HeaderA{toxval.load.skin.eye}{Load the Skin eye data}{toxval.load.skin.eye}
%
\begin{Description}\relax
Load the Skin eye data
\end{Description}
%
\begin{Usage}
\begin{verbatim}
toxval.load.skin.eye(toxval.db, source.db, verbose = F)
\end{verbatim}
\end{Usage}
%
\begin{Arguments}
\begin{ldescription}
\item[\code{toxval.db}] Database version

\item[\code{verbose}] if TRUE, print diagnostic messages along the way
\end{ldescription}
\end{Arguments}
\inputencoding{utf8}
\HeaderA{toxval.load.source\_chemical}{Perform the DSSTox mapping}{toxval.load.source.Rul.chemical}
%
\begin{Description}\relax
Perform the DSSTox mapping
\end{Description}
%
\begin{Usage}
\begin{verbatim}
toxval.load.source_chemical(toxval.db, source.db, source = NULL, verbose = T)
\end{verbatim}
\end{Usage}
%
\begin{Arguments}
\begin{ldescription}
\item[\code{toxval.db}] The version of toxvaldb to use.

\item[\code{source.db}] The source database version

\item[\code{source}] The source to update for

\item[\code{verbose}] If TRUE, print out extra diagnostic messages
\end{ldescription}
\end{Arguments}
\inputencoding{utf8}
\HeaderA{toxval.load.source\_chemical.echa\_iuclid}{Perform the DSSTox mapping}{toxval.load.source.Rul.chemical.echa.Rul.iuclid}
%
\begin{Description}\relax
Perform the DSSTox mapping
\end{Description}
%
\begin{Usage}
\begin{verbatim}
toxval.load.source_chemical.echa_iuclid(
  toxval.db,
  source.db,
  source = "ECHA IUCLID",
  verbose = T,
  chem_source
)
\end{verbatim}
\end{Usage}
%
\begin{Arguments}
\begin{ldescription}
\item[\code{toxval.db}] The version of toxvaldb to use.

\item[\code{source.db}] The source database version

\item[\code{source}] The source to update for

\item[\code{verbose}] If TRUE, print out extra diagnostic messages

\item[\code{chem\_source}] The source\_table name - this is the soruce in chemical source
\end{ldescription}
\end{Arguments}
\inputencoding{utf8}
\HeaderA{toxval.load.species}{Load the species table}{toxval.load.species}
%
\begin{Description}\relax
Load the species table
\end{Description}
%
\begin{Usage}
\begin{verbatim}
toxval.load.species(toxval.db, date_string = "2023-02-14")
\end{verbatim}
\end{Usage}
%
\begin{Arguments}
\begin{ldescription}
\item[\code{toxval.db}] The version of toxval into which the tables are loaded.

\item[\code{date\_string}] The date string for the dictionary files
\end{ldescription}
\end{Arguments}
\inputencoding{utf8}
\HeaderA{toxval.load.test}{Load TEST data from toxval\_source to toxval}{toxval.load.test}
%
\begin{Description}\relax
Load TEST data from toxval\_source to toxval
\end{Description}
%
\begin{Usage}
\begin{verbatim}
toxval.load.test(toxval.db, source.db, log = F)
\end{verbatim}
\end{Usage}
%
\begin{Arguments}
\begin{ldescription}
\item[\code{toxval.db}] The version of toxval into which the tables are loaded.

\item[\code{source.db}] The source database to use.

\item[\code{log}] If TRUE, send output to a log file
\end{ldescription}
\end{Arguments}
\inputencoding{utf8}
\HeaderA{toxval.load.toxrefdb2.1}{Load ToxRefdb data to toxval}{toxval.load.toxrefdb2.1}
%
\begin{Description}\relax
Load ToxRefdb data to toxval
\end{Description}
%
\begin{Usage}
\begin{verbatim}
toxval.load.toxrefdb2.1(toxval.db, source.db, log = F, do.init = T)
\end{verbatim}
\end{Usage}
%
\begin{Arguments}
\begin{ldescription}
\item[\code{toxval.db}] The version of toxval into which the tables are loaded.

\item[\code{log}] If TRUE, send output to a log file

\item[\code{do.init}] if TRUE, read the data in from the toxrefdb database and set up the matrix

\item[\code{verbose}] Whether the loaded rows should be printed to the console.
\end{ldescription}
\end{Arguments}
\inputencoding{utf8}
\HeaderA{toxval.load.tri}{Load TRI data data to toxval}{toxval.load.tri}
%
\begin{Description}\relax
Load TRI data data to toxval
\end{Description}
%
\begin{Usage}
\begin{verbatim}
toxval.load.tri(toxval.db, source.db, log = F)
\end{verbatim}
\end{Usage}
%
\begin{Arguments}
\begin{ldescription}
\item[\code{toxval.db}] The version of toxval into which the tables are loaded.

\item[\code{log}] If TRUE, send output to a log file

\item[\code{verbose}] Whether the loaded rows should be printed to the console.

\item[\code{do.init}] if TRUE, read the data in from the file and set up the matrix
\end{ldescription}
\end{Arguments}
\inputencoding{utf8}
\HeaderA{toxval.load.usgs\_hbsl}{Load the usgs\_hbsl (old ACToR - flex)data  from toxval source db to toxval}{toxval.load.usgs.Rul.hbsl}
%
\begin{Description}\relax
Load the usgs\_hbsl (old ACToR - flex)data  from toxval source db to toxval
\end{Description}
%
\begin{Usage}
\begin{verbatim}
toxval.load.usgs_hbsl(toxval.db, source.db, log = F)
\end{verbatim}
\end{Usage}
%
\begin{Arguments}
\begin{ldescription}
\item[\code{toxval.db}] The database version to use

\item[\code{source.db}] The source database

\item[\code{log}] If TRUE, send output to a log file
\end{ldescription}
\end{Arguments}
\inputencoding{utf8}
\HeaderA{toxval.load.ut\_hb}{Load teh uterotophic and Hershberger data}{toxval.load.ut.Rul.hb}
%
\begin{Description}\relax
Load teh uterotophic and Hershberger data
\end{Description}
%
\begin{Usage}
\begin{verbatim}
toxval.load.ut_hb(toxval.db, source.db, log = F)
\end{verbatim}
\end{Usage}
%
\begin{Arguments}
\begin{ldescription}
\item[\code{toxval.db}] The version of toxval into which the tables are loaded.

\item[\code{log}] If TRUE, send output to a log file

\item[\code{verbose}] Whether the loaded rows should be printed to the console.
\end{ldescription}
\end{Arguments}
\inputencoding{utf8}
\HeaderA{toxval.load.who\_ipcs}{Load the who\_ipcs (old ACToR - flex)data  from toxval source db to toxval}{toxval.load.who.Rul.ipcs}
%
\begin{Description}\relax
Load the who\_ipcs (old ACToR - flex)data  from toxval source db to toxval
\end{Description}
%
\begin{Usage}
\begin{verbatim}
toxval.load.who_ipcs(toxval.db, source.db, log = F)
\end{verbatim}
\end{Usage}
%
\begin{Arguments}
\begin{ldescription}
\item[\code{toxval.db}] The database version to use

\item[\code{source.db}] The source database

\item[\code{log}] If TRUE, send output to a log file
\end{ldescription}
\end{Arguments}
\inputencoding{utf8}
\HeaderA{toxval.load.wignall}{Load Wignall from toxval\_source to toxval}{toxval.load.wignall}
%
\begin{Description}\relax
Load Wignall from toxval\_source to toxval
\end{Description}
%
\begin{Usage}
\begin{verbatim}
toxval.load.wignall(toxval.db, source.db, log = F)
\end{verbatim}
\end{Usage}
%
\begin{Arguments}
\begin{ldescription}
\item[\code{toxval.db}] The version of toxval into which the tables are loaded.

\item[\code{source.db}] The version of toxval\_source from which the tables are loaded.

\item[\code{log}] If TRUE, send output to a log file
\end{ldescription}
\end{Arguments}
\inputencoding{utf8}
\HeaderA{toxval.set.mw}{Set the molecular weight in the toxval table, for use in unit conversions}{toxval.set.mw}
%
\begin{Description}\relax
Set the molecular weight in the toxval table, for use in unit conversions
\end{Description}
%
\begin{Usage}
\begin{verbatim}
toxval.set.mw(toxval.db, source = NULL)
\end{verbatim}
\end{Usage}
%
\begin{Arguments}
\begin{ldescription}
\item[\code{toxval.db}] The database version to use

\item[\code{source}] The source
\end{ldescription}
\end{Arguments}
\inputencoding{utf8}
\HeaderA{toxval.summary.stats}{Generate summary statistics on the toxval database}{toxval.summary.stats}
%
\begin{Description}\relax
Generate summary statistics on the toxval database
\end{Description}
%
\begin{Usage}
\begin{verbatim}
toxval.summary.stats(toxval.db)
\end{verbatim}
\end{Usage}
%
\begin{Arguments}
\begin{ldescription}
\item[\code{toxval.db}] The version of toxval into which the tables are loaded.
\end{ldescription}
\end{Arguments}
\printindex{}
\end{document}
